% размер шрифта: 14pt
% и указываем размер страницы - у нас это A4 (a4paper)
% подключаем макеты оформления extreport или extarticle, 
% чтобы кегль выше 12pt отображался адекватно
\documentclass[14pt, a4paper]{extarticle}

%поля: 
%слева: 2.5
%справа: 1.5
%сверху: 2.5
%снизу: 2.5
\usepackage[left=2.5cm, right=1.5cm, top=2.5cm, bottom=2.5cm]{geometry}

% установка междустрочного интервала
% закомментировано, т.к. нас устраивает пока настройки по умолчанию
% \linespread{1.3}

% кодировка файла
% не должно объявляться более одного раза
% кодировку подключаем в первую очередь
% иначе могут быть весьма неприятные непонятные ошибки
\usepackage[utf8x]{inputenc}
% шрифты
\usepackage[T2A]{fontenc}
% языки
\usepackage[english,russian]{babel}

% пакеты
\usepackage{amsmath} % align
\usepackage{amssymb}
\usepackage[usenames]{color}

% мета-данные. Т.е. данные, описывающие данные.
% в текущем случае не выводятся никуда (не будут видны в pdf)
\title{Лабораторная работа №2}
\author{Выполнил: Воронов Кирилл, М8О-207Б-19}

% подключение пакетов для ссылок и для SageTeX
\usepackage{hyperref}
\usepackage{sagetex}

% размер отступа для первой строки абзаца
\setlength{\sagetexindent}{10ex}

% определяем короткий псевдоним для объявления секции с автоматической нумерацией
\renewcommand{\thesection}{\number\numexpr\value{section}-1\relax}
% для subsection, например, задаём формат записи заголовка "номер секции.номер подсекции-1"
\renewcommand{\thesubsection}{\thesection.\number\numexpr\value{subsection}-1\relax}
\renewcommand{\thesubsubsection}{\thesubsection.\number\numexpr\value{subsubsection}-1\relax}

% установка начальных значений счетчиков секций и пр.
\setcounter{secnumdepth}{1}
% \setcounter{chapter}{1}  % этого счетчика нет в шаблоне 'article'
\setcounter{section}{1}


% открываем тег документа
\begin{document}
%Обложка
% внутри файла все необходимые настройки. Вроде отключения нумерации страниц для титульного листа
\include{"TemplateParts/TitlePage"}
% Остальное содержимое
\setcounter{page}{2} % начать нумерацию с номера 2
\tableofcontents  % вставить автоматически генерируемое оглавление для файла
\section{Приведение уравнения поверхности второго рода к каноническому виду}
\textbf{Вариант 6}
\newline
$-2*y^2 + 4*y*z - 3*z^2 + 4*y + 4*z - 12$
\newline
\begin{sagesilent}
A = matrix([
    [0, 0, 0],
    [0, -2, 2],
    [0, 2, -3]
])
B = vector([0, 4, 4])
a0 = -12
\end{sagesilent}
\newline
\textbf{Матрица квадратичной формы}
\newline
$A = \sage{A}$
\newline
\newline
\textbf{Коэффициенты линейной формы}
\newline
$B = \sage{B}$
\newline
$a_0 = \sage{a0}$
\newline
\newline
\textbf{Характеристическое уравнение}
\begin{sagesilent}
var('l', latex_name = r"\lambda")
equation_matrix = A - l * identity_matrix(3)
det = equation_matrix.det()
det = det.simplify_full()
\end{sagesilent}
\newline
$\sage{det} = 0$
\newline
\newline
\textbf{Собственные значения}
\begin{sagesilent}
lambdas = []
for i in solve(det == 0, l):
    lambdas.append(i.rhs())
\end{sagesilent}
\newline
$\lambda_{0} = \sage{lambdas[0].n(digits = 5)}$
\newline
$\lambda_{1} = \sage{lambdas[1].n(digits = 5)}$
\newline
$\lambda_{2} = \sage{lambdas[2].n(digits = 5)}$
\newline
\newline
\textbf{Собственные векторы}
\newline
\begin{sagesilent}
vectors = []
for i in range(len(lambdas)):
    vectors.append(vector(RR, *(A-lambdas[i] * identity_matrix(3)).right_kernel().basis()))
\end{sagesilent}
$s_{0} = \sage{vectors[0].n(digits = 5)}$
\newline
$s_{1} = \sage{vectors[1].n(digits = 5)}$
\newline
$s_{2} = \sage{vectors[2].n(digits = 5)}$
\newline
\newline
\textbf{Матрица перехода из нормированых собственных векторов}
\newline
\begin{sagesilent}
S = list()
for i in range(len(vectors)):
    S.append(vectors[i] / sqrt(vectors[i] * vectors[i]))
S = matrix(S).T
\end{sagesilent}
$S = \sage{S.n(digits=5)}$
\newline
\newline
\textbf{Диагональная матрица}
\newline
\begin{sagesilent}
D = S.T * A * S
\end{sagesilent}
$S^T * A * S = \sage{D.n(digits = 5)}$
\newline
\newline
\textbf{Преобразование коэффициентов линейной формы}
\newline
\begin{sagesilent}
B1 = S.T * B
\end{sagesilent}
$B' = S^T * B = \sage{B1.n(digits=5)}$
\newline
\newline
\newline
\newline
\textbf{Почти приведенное уравнение}
\newline
\begin{sagesilent}
almost = a0
var("x","y","z")
bas = [x, y, z]
for i in range(len(bas)):
    almost += D[i][i] * bas[i] ^ 2 + B1[i] * bas[i]
\end{sagesilent}
$\sage{almost == 0}$
\newline
\newline
\textbf{Приведенное уравнение}
\newline
\begin{sagesilent}
res = 0
a1 = a0
for i in range(len(bas)):
    if D[i][i] != 0:
        res += D[i][i] * (bas[i] + (B1[i]/2) / D[i][i])^2
        a1 -= D[i][i] * ((B1[i]/2) / D[i][i])^2
    else:
        res += B1[i] * bas[i]
res = (res + a1) / a1
\end{sagesilent}
$\sage{res == 0}$
\newline
\newline
\textbf{Исходный график}
\newline
\newline
\begin{sagesilent}
u(x, y, z) = -2*y^2 + 4*y*z - 3*z^2 + 4*y + 4*z - 12
\end{sagesilent}
\begin{center}
  \sageplot[trim=100 100 100 100, clip, width=10cm][png]{implicit_plot3d(u(x,y,z), (x, -10, 10), (y, -10, 10), (z, -10, 10), plot_points = 100).rotateZ(0.5)}
\end{center}
\pagebreak 
\textbf{График канонического уравнения}
\begin{center}
  \sageplot[trim=100 100 100 100, clip, width=10cm][png]{implicit_plot3d(res, (x, -10, 10), (y, -10, 11), (z, -10, 10), plot_points = 100)}
\end{center}
\end{document}
